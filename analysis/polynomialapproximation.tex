\documentclass[12pt]{article}
\usepackage{geometry}                % See geometry.pdf to learn the layout options. There are lots.
\geometry{letterpaper}                   % ... or a4paper or a5paper or ... 
%\geometry{landscape}                % Activate for for rotated page geometry
%\usepackage[parfill]{parskip}    % Activate to begin paragraphs with an empty line rather than an indent
\usepackage{graphicx}
\usepackage{amsmath,amssymb,amsfonts,amsthm} 
\usepackage{epstopdf}

% Computer Concrete
%\usepackage{concmath}
%\usepackage[T1]{fontenc}

% Times variants
%
%\usepackage{mathptmx}
%\usepackage[T1]{fontenc}
%
%\usepackage[T1]{fontenc}
%\usepackage{stix}
%
% Needs to typeset using LuaLaTeX:
%\usepackage{unicode-math}
%\setmainfont{XITS}
%\setmathfont{XITS Math}

\DeclareGraphicsRule{.tif}{png}{.png}{`convert #1 `dirname #1`/`basename #1 .tif`.png}

\theoremstyle{plain}
\newtheorem{theorem}{Theorem}
\newtheorem{corollary}[theorem]{Corollary}
\newtheorem{lemma}[theorem]{Lemma}
\newtheorem{proposition}[theorem]{Proposition}
\newtheorem{conjecture}[theorem]{Conjecture}
\newtheorem{question}[theorem]{Question}

\theoremstyle{definition}
\newtheorem{definition}[theorem]{Definition}
\newtheorem{example}[theorem]{Example}
\newtheorem*{keywords}{Keywords}

\theoremstyle{remark}
\newtheorem{remark}[theorem]{Remark}
\newtheorem{note}[theorem]{Note}

\title{Polynomial Approximation and Sequences of Functions}
\author{N. Trong}
\date{\today}                                           % Activate to display a given date or no date

\begin{document}
\maketitle

\begin{theorem}[Taylor's Theorem]
If $f', \ldots, f^{(n+1)}$ are defined on $[a, x]$, then 
$$f(x) = f(a) + f'(a)(x - a) + \cdots + \frac{f^{(n)}(a)}{n!}(x - a)^n + R_n(x)$$
where $R_n(x) = \frac{f^{(n+1)}(t)}{(n+1)!}(x - a)^{n+1}$ for some $t$ in $(a, x)$.
\end{theorem}

\begin{note}
The Mean Value Theorem is a special case of Taylor's Theorem:
$$f(b) = f(a) + f'(c)(b - a)$$
for some $c$ between $a$ and $b$.
\end{note}

\begin{theorem}
Uniform convergence of functions preserves continuity, i.e. if $f_n$ are continuous and approach $f$ uniformly, then $f$ is continuous.
\end{theorem}

\begin{question}
What about differentiability, i.e. if $f_n$ are differentiable and approach $f$ uniformly, is $f$ always differentiable, and is $\lim f_n' = f'$? 
\end{question}

\begin{example}
No to the second question: the functions $f_n(x) = \frac{1}{n} \sin(nx)$ converge uniformly to the zero function, which \textit{is} differentiable. But, the limit of the derivatives don't exist. What about just differentiability? 
\end{example}



\begin{keywords}
Taylor's Theorem, Taylor polynomial, error / remainder term, Cauchy, Lagrange, integral form, point-wise, uniform convergence, metric space, Cauchy criterion, Koch snowflake.
\end{keywords}

\end{document}
